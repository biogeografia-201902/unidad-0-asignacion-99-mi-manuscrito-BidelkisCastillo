\documentclass[11pt,]{article}
\usepackage[left=1in,top=1in,right=1in,bottom=1in]{geometry}
\newcommand*{\authorfont}{\fontfamily{phv}\selectfont}
\usepackage[]{mathpazo}


  \usepackage[T1]{fontenc}
  \usepackage[utf8]{inputenc}



\usepackage{abstract}
\renewcommand{\abstractname}{}    % clear the title
\renewcommand{\absnamepos}{empty} % originally center

\renewenvironment{abstract}
 {{%
    \setlength{\leftmargin}{0mm}
    \setlength{\rightmargin}{\leftmargin}%
  }%
  \relax}
 {\endlist}

\makeatletter
\def\@maketitle{%
  \newpage
%  \null
%  \vskip 2em%
%  \begin{center}%
  \let \footnote \thanks
    {\fontsize{18}{20}\selectfont\raggedright  \setlength{\parindent}{0pt} \@title \par}%
}
%\fi
\makeatother




\setcounter{secnumdepth}{3}



\title{Las Hormigas del Tali: Sol, cansancio y dolor, viviendo la
experiencia.\\
Hormigas Talibanesas\\
Aquí todos somos Talibanes  }



\author{\Large Bidelkis A. Castillo S. (Dora)\vspace{0.05in} \newline\normalsize\emph{Estudiante, Universidad Autónoma de Santo Domingo (UASD)}  }


\date{}

\usepackage{titlesec}

\titleformat*{\section}{\normalsize\bfseries}
\titleformat*{\subsection}{\normalsize\itshape}
\titleformat*{\subsubsection}{\normalsize\itshape}
\titleformat*{\paragraph}{\normalsize\itshape}
\titleformat*{\subparagraph}{\normalsize\itshape}

\titlespacing{\section}
{0pt}{36pt}{0pt}
\titlespacing{\subsection}
{0pt}{36pt}{0pt}
\titlespacing{\subsubsection}
{0pt}{36pt}{0pt}





\newtheorem{hypothesis}{Hypothesis}
\usepackage{setspace}

\makeatletter
\@ifpackageloaded{hyperref}{}{%
\ifxetex
  \PassOptionsToPackage{hyphens}{url}\usepackage[setpagesize=false, % page size defined by xetex
              unicode=false, % unicode breaks when used with xetex
              xetex]{hyperref}
\else
  \PassOptionsToPackage{hyphens}{url}\usepackage[unicode=true]{hyperref}
\fi
}

\@ifpackageloaded{color}{
    \PassOptionsToPackage{usenames,dvipsnames}{color}
}{%
    \usepackage[usenames,dvipsnames]{color}
}
\makeatother
\hypersetup{breaklinks=true,
            bookmarks=true,
            pdfauthor={Bidelkis A. Castillo S. (Dora) (Estudiante, Universidad Autónoma de Santo Domingo (UASD))},
             pdfkeywords = {Formicidae, insectos, especies, Hymenoptera, hispaniola},  
            pdftitle={Las Hormigas del Tali: Sol, cansancio y dolor, viviendo la
experiencia.\\
Hormigas Talibanesas\\
Aquí todos somos Talibanes},
            colorlinks=true,
            citecolor=blue,
            urlcolor=blue,
            linkcolor=magenta,
            pdfborder={0 0 0}}
\urlstyle{same}  % don't use monospace font for urls

% set default figure placement to htbp
\makeatletter
\def\fps@figure{htbp}
\makeatother

\usepackage{pdflscape} \newcommand{\blandscape}{\begin{landscape}}
\newcommand{\elandscape}{\end{landscape}}


% add tightlist ----------
\providecommand{\tightlist}{%
\setlength{\itemsep}{0pt}\setlength{\parskip}{0pt}}

\begin{document}
	
% \pagenumbering{arabic}% resets `page` counter to 1 
%
% \maketitle

{% \usefont{T1}{pnc}{m}{n}
\setlength{\parindent}{0pt}
\thispagestyle{plain}
{\fontsize{18}{20}\selectfont\raggedright 
\maketitle  % title \par  

}

{
   \vskip 13.5pt\relax \normalsize\fontsize{11}{12} 
\textbf{\authorfont Bidelkis A. Castillo S. (Dora)} \hskip 15pt \emph{\small Estudiante, Universidad Autónoma de Santo Domingo (UASD)}   

}

}








\begin{abstract}

    \hbox{\vrule height .2pt width 39.14pc}

    \vskip 8.5pt % \small 

\noindent Mi resumen


\vskip 8.5pt \noindent \emph{Keywords}: Formicidae, insectos, especies, Hymenoptera, hispaniola \par

    \hbox{\vrule height .2pt width 39.14pc}



\end{abstract}


\vskip 6.5pt


\noindent  \section{Introducción}\label{introducciuxf3n}

El mundo que habitamos está compuesto de una gran diversidad de
animales, en la mayoria de los casos los humanos solo prestan atención y
estudian a aquellos que son perceptibles a la visión, pero dentro de
este gran grupo de vida animal se encuentran unos pequeños y en
ocasiones muy diminutos seres que realizan una labor importante en el
Ecosistema: \textbf{``Los Insectos''}.

Según Peña (1996), los insectos son animales como los demás, números y
ricos en especie, con una densidad poblacional de más de 750,000
conocidas y las que faltan por encontrar.

Dentro de este gran número de especie se encuentran las hormigas, éstas
pertenecen al grupo de las Hymenoptera y a la familia Formicidae. Según
Fernández \& Sendoya (2004) las hormigas son insectos visibles, famosas
y sobresalientes, e imperiosas en la tierra. Se dice que estas son muy
abundantes en los ecosistemas tropicales, disminuyendo hacia las
latitudes templadas; mientras que en países de climas extremos pueden
ser localmente abundantes (como las grandes colonias de Formica) y
decisivas. Continúa Peña (1996) diciendo que en el presente hay cerca de
11500 especies de hormigas definidas en todo el Mundo, de las cuales
unas 3100 corresponden a la Región Neotropical. Con casi un 30\% de las
hormigas del Mundo, el Neotrópico es la región más rica en hormigas.

Las hormigas juegan un papel muy impotante en el ecosistema, siendo
éstas dispersadoras de semillas que caen de la planta al suelo, son
capaces de polinizar cargando el polen a plantas cercanas, son
depredadoras y descomponedoras e incrementan la materia orgánica.

Segun Lubertazzi (2019) en la isla de la hispaniola, que comprende Haití
y República Dominicana las hormigas estan representada por 9 familias,
43 géneros, 137 especies, y 10 subespecies, de las cuales 126 especies
son nativas y 11 introducidas, 60 de las especies nativas son endémicas
y 68 se encuentran en ambos países, de estas 99 son de Haiti y 126 de
República Dominicana.

Pregunta 1. ¿Las hormigas se encuentran en mayor concentración en áreas
aseadas y sin basura o en zonas donde hay zafacones ?

Pregunta 2. ¿Se relaciona la falta de vegetación herbácea con la
abundancia de las hormigas ?

Este estudio lo que buscar es identificar hasta género las Formicidae
(hormigas) que se encuentran distribuidas en el campus de la Universidad
Autónoma de Santo Domingo (UASD).

\section{Metodología}\label{metodologuxeda}

Listado de intrumentos utilizados Días, tiempo de recolecta, influencia
de pavimento Formulario, identificación (proceso) Atún y la influencia a
las hormigas

\section{Resultados}\label{resultados}

\ldots

\section{Discusión}\label{discusiuxf3n}

\section{Agradecimientos}\label{agradecimientos}

\section{Información de soporte}\label{informaciuxf3n-de-soporte}

\ldots

\section{\texorpdfstring{\emph{Script}
reproducible}{Script reproducible}}\label{script-reproducible}

\ldots

\section*{Referencias}\label{referencias}
\addcontentsline{toc}{section}{Referencias}

\hypertarget{refs}{}
\hypertarget{ref-fernandez2004lista}{}
Fernández, F., \& Sendoya, S. (2004). Lista sinonímica de las hormigas
neotropicales (hymenoptera: Formicidae). \emph{Biota Colombiana},
\emph{5}(1).

\hypertarget{ref-lubertazzi2019ants}{}
Lubertazzi, D. (2019). THE ants of hispaniola. \emph{Bulletin of the
Museum of Comparative Zoology}, \emph{162}(2), 59--210.

\hypertarget{ref-pena1996introduccion}{}
Peña, L. E. (1996). \emph{Introducción al estudio de los insectos de
chile}. Editorial Universitaria.




\newpage
\singlespacing 
\end{document}
